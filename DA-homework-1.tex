\documentclass{article}
\usepackage[T1]{fontenc}
\usepackage{amssymb, amsmath, graphicx, subfigure}
\usepackage{algorithm2e}

\setlength{\oddsidemargin}{.25in}
\setlength{\evensidemargin}{.25in}
\setlength{\textwidth}{6in}
\setlength{\topmargin}{-0.4in}
\setlength{\textheight}{8.5in}

\newcommand{\heading}[6]{
  \renewcommand{\thepage}{#1-\arabic{page}}
  \noindent
  \begin{center}
  \framebox{
    \vbox{
      \hbox to 5.78in { \textbf{#2} \hfill #3 }
      \vspace{4mm}
      \hbox to 5.78in { {\Large \hfill #6  \hfill} }
      \vspace{2mm}
      \hbox to 5.78in { \textit{Instructor: #4 \hfill #5} }
    }
  }
  \end{center}
  \vspace*{4mm}
}

\newtheorem{theorem}{Theorem}
\newtheorem{definition}[theorem]{Definition}
\newtheorem{remark}[theorem]{Remark}
\newtheorem{lemma}[theorem]{Lemma}
\newtheorem{corollary}[theorem]{Corollary}
\newtheorem{proposition}[theorem]{Proposition}
\newtheorem{claim}[theorem]{Claim}
\newtheorem{observation}[theorem]{Observation}
\newtheorem{fact}[theorem]{Fact}
\newtheorem{assumption}[theorem]{Assumption}

\newenvironment{proof}{\noindent{\bf Proof:} \hspace*{1mm}}{
	\hspace*{\fill} $\Box$ }
\newenvironment{proof_of}[1]{\noindent {\bf Proof of #1:}
	\hspace*{1mm}}{\hspace*{\fill} $\Box$ }
\newenvironment{proof_claim}{\begin{quotation} \noindent}{
	\hspace*{\fill} $\diamond$ \end{quotation}}

\newcommand{\problemset}[3]{\heading{#1}{236755: Distributed Algorithms}{#2}{Hagit Attiya}{#3}{Home Assignment #1}}



%%%%%%%%%%%%%%%%%%%%%%%%%%%%%%%%%%%%%%%%%%%%%%%%%%%%%%%%%%%%%%%%%%%%%%%%%%%%%%%
% PLEASE MODIFY THESE FIELDS AS APPROPRIATE
\newcommand{\problemsetnum}{1}          % problem set number
\newcommand{\duedate}{July 7, 2024}  % problem set deadline
\newcommand{\studentname}{May Shimony \& Michael Leibman}    % full name of student (i.e., you)
% PUT HERE ANY PACKAGES, MACROS, etc., ADDED BY YOU
% ...
%%%%%%%%%%%%%%%%%%%%%%%%%%%%%%%%%%%%%%%%%%%%%%%%%%%%%%%%%%%%%%%%%%%%%%%%%%%%%%%


%%%%%%%%%%%%%%%%%%%%%%%%%%%%%%%%%%%%%%%%%%%%%%%%%%%%%%%%%%%%%%%%%%%%%%%%%%%%%%%
\begin{document}
\problemset{\problemsetnum}{\duedate}{\studentname}

\subsection*{Question 1}

The statement is correct. We'll prove that Algorithm 1 provides the
same properties - mutual exclusion, bounded waiting and FIFO order
- as the algorithm shown in class.

\uline{Mutual Exclusion:}

Assume by contradiction that 2 processes, $p_{i}$ and $p_{j}$, are
in the critical section at the same time in configuration C of an
execution $\alpha$. Therefore, both processes have executed lines
2 and 3 in the algorithm and have achieved the condition in line 4
- $curr\_first$ is both $position\_p_{i}$ and $position\_p_{j}$.

$Position\_p_{i}$ was obtained in line 1 by $rmw\left(Last,f\right)$,
which, being an atomic operation, provides a unique position. $Position\_p_{j}$
was obtained in the same way. Therefore, $position\_p_{i}\neq position\_p_{j}$.

While waiting to enter the critical section, during lines 2 and 3,
the operation $rmw$ returns the current first position, and again,
due to $rmw$ being atomic, the returned value cannot be different
when different processes receive it at the same time - so $p_{i}$
and $p_{j}$ are both aware of the current first process in line to
enter the critical section.

Since $position\_p_{i}\neq position\_p_{j}$, it's not possible for
both $position\_p_{i}$ and $position\_p_{j}$ to be of value to $curr\_first$
and for both processes to enter the critical section at the same time,
in contradiction to the assumption.

$\Rightarrow$ Algorithm 1 provides mutual exclusion.

\uline{Bounded Waiting:}

Assume by contradiction that in configuration C of an execution $\alpha$,
there is no bound on the number of times processes enter the critical
section while process $p$ is waiting to enter it.

Process $p$ has obtained $position\_p$ in line 1 by $rmw\left(Last,f\right)$.

Since the wait is unbounded, an infinite number of processes enter
the critical section before $p$. Therefore an infinite number of
processes have also left the critical section and performed the exit
code in line 5, incrementing the shared variable $First$ by 1.

However, $First$ is incremented using the function $f$, incrementing
modulo 1. So after $n$ increments - $n$ processes that have left
the critical section - $First$ will start over the count and cycle
through all $n$ possible positions. The shared variable $Last$ is
also incremented using the function $f$, meaning it will also cycle
through positions $0$ to $n-1$, and therefore $position\_p$ is
bound by these values. So $First$ will reach $position\_p$ after
at most $n-1$ processes have entered the critical section, in contradiction
to the assumption.

$\Rightarrow$ Algorithm 1 provides bounded waiting.

\uline{FIFO Order:}

Assume by contradiction that in configuration C of an execution $\alpha$,
process $p_{i}$ has requested to enter the critical section before
$p_{j}$ requested the same, and process $p_{j}$ entered the critical
section before process $p_{i}$.

Process $p_{i}$ obtained $position\_p_{i}$ in line 1 and so has
$p_{j}$. Since the operation $rmw$ is atomic and the function $f$
increments by 1, and due to the assumption that $p_{i}$ applied the
entry code first, $position\_p_{i}<position\_p_{j}$.

Process $p_{j}$ then enters the critical section, which it can do
after satisfying the condition in line 4 - $curr\_first==position\_pj$.

As mentioned previously, $curr\_first$ is obtained by the incrementation
of the shared variable $First$ by 1 in a way that ensures the value
is identical for different processes reading at the same time. Therefore,
for $curr\_first$ to equal $position\_p_{j}$ before it reaches $position\_p_{i}$,
it would have to reach the value of $position\_p_{j}$ before reaching
the value of $position\_p_{i}$ - contradicting the assumption $position\_p_{i}<position\_p_{j}$.

$\Rightarrow$ Algorithm 1 provides FIFO order.

\subsection*{Question 2}


\subsection*{Question 3}

Assume by contradiction that in a no-deadlock, mutual exclusion algorithm
that only uses read-write registers, the entry code for a process
$p$ \emph{doesn't} write to some shared variable $A$, followed by
a read to a different shared variable $B$ without the process writing
to $B$ in between.

We will show how, assuming each of the four conditions mentioned are
not occurring, a violation of the no-deadlock or mutual exclusion
terms ensues.

\uline{Process \mbox{$p$} doesn't write to shared variable \mbox{$A$}
in the entry code:}

Given the algorithm uses only read-write registers, in order for different
processes to determine the validity of entering the critical section,
a process has to notify them through these registers of its intent
to enter the critical section. If $p$ doesn't write to a shared variable
in the entry code, other processes won't have a way of knowing $p$
wants to enter or is in the critical section, resulting in a possible
violation of mutual exclusion if several processes enter the critical
section at the same time.

\uline{Process \mbox{$p$} doesn't read from shared variable \mbox{$B$}
in the entry code:}

Using the same logic previously applied, a process wanting to enter
the critical section needs to verify that no other processes are in
the critical section before entering it. If $p$ doesn't read from
a shared variable, it has no way of knowing about other processes
entering the critical section themselves, which also results in a
possible violation of mutual exclusion if several processes enter
the critical section simultaneously.

\uline{Process p reads and writes to the same shared variable \mbox{$A$}
in entry code:}

Assuming the entry code for the critical section contains a write
and a read from the same variable, while in the entry code, $p$ will
write to $A$ stating its intent to enter the critical section and
later read from $A$ about its own intentions to enter. This could
result in deadlock due to process $p$ waiting for the critical section
to be empty, not realizing the process ``occupying'' the critical
section is itself.

Alternatively, depending on the specific value $p$ writes to $A$,
it is possible that $p$ realizes the process entering the critical
section is $p$ itself and therefore enters the critical section.
However, in that case, it is possible another process $q$ reads from
the shared variable $A$ while process $p$ writes to it. The value
$q$ reads from $A$ may indicate it can enter the critical section.
Consequently, while $p$ enters the critical section, $q$ writes
to $A$ of its intent to enter the critical section and assumes it
can do so, overwriting the value written by $p$. This will violate
mutual exclusion since $q$ will enter the critical section at the
same time $p$ is already in there.

\uline{Process \mbox{$p$} writes to shared variable \mbox{$A$},
writes to shared variable \mbox{$B$} and reads from shared variable
\mbox{$B$} in entry code:}

Assuming the stated order of actions, the following scenario is possible:
Process $p$, wanting to enter the critical section, writes to shared
variables $A$ and $B$. Then, while reading from $B$, the process
is reading its own written value - therefore, $p$ hasn't actually
received information from other processes about whether or not it
can enter the critical section. Similarly to the previous part of
the proof, this can result in mutual exclusion violations and create
deadlocks if writing to $B$ prevents process $p$ and other processes
from entering the critical section at all.

\subsection*{Question 4}



% Put bibliography here (if any).
% Example:
%
%\begin{thebibliography}{99}
%
% \bibitem[CLRS]{CLRS}
% Thomas H. Cormen, Charles E. Leiserson, Ronald L. Rivest, and Clifford Stein,
% \emph{Introduction to Algorithms}, Second Edition.
% MIT Press and McGraw-Hill, 2001.
%
%\end{thebibliography}

\end{document}
%%%%%%%%%%%%%%%%%%%%%%%%%%%%%%%%%%%%%%%%%%%%%%%%%%%%%%%%%%%%%%%%%%%%%%%%%%%%%%%
