%% LyX 2.3.7 created this file.  For more info, see http://www.lyx.org/.
%% Do not edit unless you really know what you are doing.
\documentclass[american]{article}
\usepackage{fontspec}
\setmainfont[Mapping=tex-text]{David}
\setsansfont[Mapping=tex-text]{David}
\setmonofont{David}
\usepackage{geometry}
\geometry{verbose,tmargin=2cm,bmargin=3cm,lmargin=1cm,rmargin=1cm,headheight=0cm,headsep=0cm,footskip=2cm}

\makeatletter
\@ifundefined{date}{}{\date{}}
%%%%%%%%%%%%%%%%%%%%%%%%%%%%%% User specified LaTeX commands.
\newcommand{\linethrough}{\mathpalette\@thickbar}
\newcommand{\@thickbar}[2]{{#1\mkern0mu\vbox{
    \sbox\z@{$#1#2\mkern-1.5mu$}%
    \dimen@=\dimexpr\ht\tw@-\ht\z@+2\p@\relax % The +2 represents the vertical shift of the line.
    \hrule\@height0.5\p@ % The 0.5 represent the thickness on the line.
    \vskip\dimen@
    \box\z@}}
}
\makeatother
\newcommand{\mathstrike}[1]{\ensuremath{\linethrough{#1}}}

\makeatother

\usepackage{polyglossia}
\setdefaultlanguage[variant=american]{english}
\begin{document}

\section{Preliminaries}

In this section, we will outline the model assumptions and cryptographic
techniques that form the basis for Byzantine broadcast protocols in
adversarial settings.

\subsection{Byzantine Fault Model}

The Byzantine fault model assumes that up to $t$ processes in a network
of $n$ processes may behave arbitrarily, due to failures or malicious
intent. The faulty processes can deviate from the protocol in any
way and, if malicious, may conspire with other corrupt processes to
disrupt the consensus the reliable processes aspire to achieve. In
the BA protocols covered in class, consensus was reached under the
assumption that $n>3t$, meaning honest nodes constitute a majority
- but in the dishonest majority model , corrupt nodes can outvote
the honest ones and as a result BA protocols often rely on cryptographic
methods to provide guarantees of correctness and security.

\subsection{The Dishonest Majority Model}

In the dishonest majority model, we assume that our adversary can
corrupt a majority of processes in the network, rather than a minority
as is the case in classic BA problems. When $t\geq\frac{n}{2}$, malicious
processes have the ability to conspire and coordinate attacks that
overwhelm the votes cast by honest processes and therefore wildly
disrupt the consensus attempt, and BA protocols struggle to overcome
that threat. Since securing the bulk of the votes can no longer be
a dependable way of reaching agreement, BA protocols under a dishonest
majority model need to reach consensus even when reliable processes
are a minority - so a way of detecting which processes \emph{are}
honest is crucial.

\subsection{Cryptographic Methods and Assumptions}

In order to address the challenges posed by a dishonest majority,
we will use the following cryptographic tools to ensure security and
reliability.

\subsubsection{Digital Signatures}

Digital signatures are a common way to approach ensuring message integrity,
where each process digitally adds its signature when sending messages.
We assume a trusted public-key infrastructure (PKI) is used, where
each process has a unique public/private key pair that was fairly
generated and distributed, and is used to verify the validity of the
signatures added by the different processes, and the public key is
incorruptible so that the actual verification is correct. By using
digital signatures we can guarantee that malicious processes cannot
forge messages on behalf of honest nodes.

\subsubsection{Verifiable Random Functions}

A verifiable random function (VRFs) is a cryptographic primitive that
provides a deterministic way to produce a random output while allowing
anyone to verify that the output was correctly computed. We assume
that the primitive can be trusted upon to be correct and reliable
without the need to trust another application to provide randomness
in the protocol.

\subsubsection{Communication Model}

The communication model considered is \emph{synchronous}. The protocols
in a synchronous communication model proceed in rounds, where in each
round every process can send a message to every other process in an
authenticated way and these messages are guaranteed to arrive within
some maximum network delay. Messages from honest processes cannot
be dropped and will be delivered. While adversaries can corrupt processes
during a round, they cannot corrupt messages that were sent before
the corruption occurred.
\end{document}
